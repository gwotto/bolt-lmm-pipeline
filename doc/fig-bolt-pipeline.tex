\documentclass[border=0.5cm]{standalone}

\usepackage[latin1]{inputenc}
\usepackage{tikz}
\usetikzlibrary{shapes, shapes.misc, arrows, positioning}
\usetikzlibrary{backgrounds,fit,calc}

\begin{document}
\pagestyle{empty}

\tikzset{
  background1/.style={
    draw,
    %fill=yellow!30,
    align=left
  }}


%% block styles

\tikzstyle{processblock} = [rectangle, draw=gray, fill=blue!5,
minimum width=5em, text centered, rounded corners, minimum height=4em]

\tikzstyle{datablock} = [draw=gray, chamfered rectangle, chamfered rectangle corners=north east, minimum height=2em]

\tikzstyle{line} = [draw, -latex']



\begin{tikzpicture}

        
  \node [processblock] (init) {initialize-pipeline.py};

  \node [processblock, below= of init] (main1) {main.py};

  \node [datablock, left= of main1] (config) {config.yml};

  
  \node [datablock, right= of main1] (samples) {samples-file};
  \node [datablock, above= 0.2cm of samples] (pheno) {pheno-file};
  \node [datablock, below= 0.2cm of samples] (fam) {fam-file};
  
  \node [processblock, below=1.5cm of main1] (plink2) {run-plink.py};

  \node [processblock, left= of plink2] (plink1) {run-plink.py};

  \node [processblock, right= of plink2] (plink3) {run-plink.py};

  \node [processblock, below=1.5cm of plink2] (main2) {main.py};


  \node [processblock, below=1.5cm of main2] (bolt2) {run-bolt.py};

  \node [processblock, left= of bolt2] (bolt1) {run-bolt.py};

  \node [processblock, right= of bolt2] (bolt3) {run-bolt.py};


  \node [processblock, below=1.5cm of bolt2] (main3) {main.py};

  \node [datablock, below= of main3] (output) {bolt-lmm output};

  
  % \node [outputblock, below= of pipeline1] (vcf1) {sample-1.vcf};
  % \node [outputblock, below= of pipeline2] (vcf2) {sample-2.vcf};
  % \node [outputblock, below= of pipeline3] (vcf3) {sample-3.vcf};

  \path [line] (init) edge
    (main1);
  
  \path [line] (config) edge
    (main1);

    \path [line] (pheno) edge
    (main1);

    \path [line] (samples) edge
    (main1);
    
    \path [line] (fam) edge
    (main1);
    
    \path [line] (main1) edge
    node[left, align=center](chromosome1) {chr\_1}
    (plink1);

    \path [line] (main1) edge
    node[left, align=center](chromosome2) {chr\_2}
    (plink2);

    \path [line] (main1) edge
    node[left, align=center](chromosome3) {chr\_n}
    (plink3);

    \path [line] (plink1) edge
    % node[left, align=center](chromosome1-1) {chr 1}
    (main2);

    \path [line] (plink2) edge
    % node[left, align=center](chromosome1-1) {chr 1}
    (main2);

    \path [line] (plink3) edge
    % node[left, align=center](chromosome1-1) {chr 1}
    (main2);


    \path [line] (main2) edge
    node[left, align=center](chunk1) {chunk\_1}
    (bolt1);

    \path [line] (main2) edge
    node[left, align=center](chunk2) {chunk\_2}
    (bolt2);

    \path [line] (main2) edge
    node[right, align=center](chunk3) {chunk\_n}
    (bolt3);

    
    \path [line] (bolt1) edge
    (main3);

    \path [line] (bolt2) edge
    (main3);

    \path [line] (bolt3) edge
    (main3);

    \path [line] (main3) edge
    % node[left, align=center](chromosome1-1) {chr 1}
    (output);

    

    % \path [line] (pipeline1) edge
    % (vcf1);

    % \path [line] (pipeline2) edge
    % (vcf2);

    % \path [line] (pipeline3) edge
    % (vcf3);


    
  \end{tikzpicture}

\end{document}

%%% Local Variables:
%%% mode: latex
%%% TeX-master: t
%%% End:
